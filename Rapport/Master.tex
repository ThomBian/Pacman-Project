\documentclass{article}
\usepackage[utf8]{inputenc}
\usepackage[T1]{fontenc}
\usepackage{lmodern}
\usepackage{graphicx}
\usepackage[french]{babel}
\usepackage{fullpage}


%def du style du code java dans latex%
\usepackage{listings}
\usepackage{color}

\definecolor{dkgreen}{rgb}{0,0.6,0}
\definecolor{gray}{rgb}{0.5,0.5,0.5}
\definecolor{mauve}{rgb}{0.58,0,0.82}

\lstset{frame=tb,
  language=Java,
  aboveskip=3mm,
  belowskip=3mm,
  showstringspaces=false,
  columns=flexible,
  basicstyle={\small\ttfamily},
  numbers=none,
  numberstyle=\tiny\color{gray},
  keywordstyle=\color{blue},
  commentstyle=\color{dkgreen},
  stringstyle=\color{mauve},
  breaklines=true,
  breakatwhitespace=true,
  tabsize=4
}



\graphicspath{{pics/}}


\author{Thomas Bianchini - Nathanael Couret - Antoine Valette - Clement Taboulot}
\title{Projet Pacman}
\date{2 mars 2015}

\begin{document}

\maketitle

\centerline{\includegraphics[scale=0.3]{pics/Pacman_HD}}

\pagebreak

\tableofcontents

\pagebreak

% Section 1 : Division du travail
\section{Organisation du projet}

Étant un groupe de quatre collaborateurs, nous avons commencé par nous organiser afin que chacun connaisse son rôle dans le but de faire avancer le projet correctement. \\
Tout d'abord, la base de notre travail repose sur la possibilité pour chacun des membres de disposer des ressources nécessaires afin d'avancer ses tâches. Pour cela, nous avons mis en place un repository git (hébergé sur GitHub). Ce choix se justifie car la mise en place est simple et l'équipe avait les connaissances suffisantes d'utilisation. \\
Ensuite nous avons décidé de réfléchir tous ensemble sur le sujet afin d'ébaucher une architecture pour notre application qui sera expliquer en détail dans la partie structure de l'application. Puis nous nous sommes mis d'accord sur les tâches de chacun :
\begin{itemize}
	\item Antoine : algorithme minimax
	\item Clement : algorithme du plus court chemin
	\item Nathanël : deplacement aleatoire, gestion des cartes pourries, gestion des victoires/défaites IHM
	\item Thomas : parser de fichier de map, mis en place de la structure, gestion IHM
	\item Clement - Thomas : deplacement des fantomes
	\item Groupe : code review, javadoc, rapport
\end{itemize}

%Section 2 : Structure de l'application
\section{Structure de l'application}

\subsection{Le Modele - Vue - Controlleur}

La première idée que nous avons eu a été d'introduire un desgin pattern afin de structurer l'architecture globale de l'application. \\ Le pattern MVC permet en effet de découper l'application en trois grandes parties :
\begin{itemize}
 	\item l'affichage de données (Vue)
 	\item la sauvegarde et manipulation de données (Modele)
 	\item la gestion des interactions de l'utilisateur (Controlleur)
 \end{itemize}
Nous avons donc adapté le MVC à notre application dont voici le diagramme de classe : \\[0.5cm]
\centerline{\includegraphics[scale=0.5]{MVC}}

En suivant la logique de ce diagramme de classe, nous remarquons que le controller ne possède que deux méthodes. En effet, nous avons à gérer deux types d'évènement : l'avancement de la simulation gérée par la méthode update() et la remise à zéro de la simulation (restart()).
Afin de détailler une des deux méthodes, prenons l'exemple de la mise à jour :

\begin{lstlisting}
/* DANS LA VUE */
public void next() {
	controller.update();
}

/* DANS LE CONTROLLEUR */
//Methode update
public void update () {
	updatePacman();
	updateGhosts();
}

//Methode updatePacman
private void updatePacman() {
	Tile tile = this.model.getPacman().getPosition();
	Tile newTilePM = this.model.movePacman();
	this.model.getPacman().setPosition(newTilePM);
	this.model.updateEntityPosition(this.model.getPacman(), tile);
	view.drawPacMan(newTilePM.getX(), newTilePM.getY());
	view.drawSpace(tile.getX(), tile.getY());
}
\end{lstlisting}
Lorsque l'utilisateur appuie sur le bouton next sur l'IHM, la méthode next() de la vue est appelée. La vue fait alors appel à son controller qui devra gérer cette évènements. \\
Dans ce cas, la méthode update() du controller a pour rôle de mettre à jour Pacman et les fantômes. En se penchant sur la mise à jour de Pacman, nous voyons le mécanisme de déplacement d'un personnage. Nous récupérons la case courante, puis le modele se charge de calculer la nouvelle position du personnage, se fait ensuite l'attribution de cette nouvelle position. La partie gestion des données est finie, il ne reste qu'à déléguer l'affichage à la vue. Les méthodes de dessin de la vue se servent de l'objet simulable. Quant au modele, le calcul des postions est délégué au personnage (voir la partie gestion de personnage ci-après). Son principal rôle est de stocker les données et de les maintenir à jour à chaque tour. Son second rôle est aussi de parser les fichiers de cartes et créer les données associées.






\subsection{La gestion des personnages}

Les personnages sont la pièce maîtresse de l'application. Pour la simulation il y a deux types de personnages : Pacman et les fantomes.
Ci dessous, le diagramme de classes lié aux personnages : \\[0.5cm]
\centerline{\includegraphics[scale=0.2]{Entity}}

Le premier choix était de savoir si Pacamn et les fantômes seraient des classes à part entière. La seule différence notable est le fait que Pacman mange tout ce qu'il trouve sur son chemin (excepté les fantomes) tandis que les fantomes doivent se déplacer sans altérer le contenu des cases qu'ils parcourent (excepté Pacman). C'est pourquoi la classe Ghost a un attribut lastContent afin de sauvegarder le contenu (cf Enum Content dans le diagramme) de la case sur laquelle le fantome se trouve.
Ceci permettra notamment de redessiner correctement la carte après passage d'un fantome. \\
D'autre part les points communs de tous les personnages sont le fait qu'ils se deplacent. Ce déplacement nécessite donc que chaque personnage possède une position, sa position courante. La classe Tile (cf partie Gestion de la carte) sert à stocker l'emplacement de nos personages. La méthode move() du personnage retourne la nouvelle position du personnage et délègue ce calcul à la stratégie (cf partie sur les Stragtégies) du personnage.



\subsection{Les stratégies de déplacement d'un personnage}

Notre projet inclut des stratégies car les personnages peuvent en effet se déplacer de différentes manières en fonction du contexte ou en fonction du choix de l'utilisateur. Comme le montre le diagramme ci après, chaque stratégie correspond à un type de déplacement (les algorithmes sont expliqués plus loin). \\[0.5cm]

\centerline{\includegraphics{}}

La méthode move() de chaque stratégie prend en paramètre un Board (cf Gestion des cartes) et une Tile représentant la position courante d'un personnage sur le jeu. La méthode retourne la nouvelle position du personnage.


\subsection{Gestion de la carte}

Le modele du MVC permet de stocker toutes les données de la simulation. Un carte est un tableau en deux dimensions de taille fixe qui contient des éléments de type Content. Ainsi au démarrage de l'application le modele est initialisé en parsant le fichier de carte sur lequel on veut jouer. Les données lues au moment du parsing sont stockées dans un objet de type Board dont voici la description : \\[0.5cm]
\centerline{\includegraphics[scale=0.2]{Map}}

Le choix du tableau de taille fixe nous permet d'autre part de gérer les fichiers malformés, les dépassement ce tableau... En effet, c'est grâce à cette objet que nous pouvons tester les cartes "pourries" (les cartes où Pacman ne pourra jamais gagner car il est bloqué).


Nous avons essayé au travers de cette application de mettre en place des mécanismes de Progammation Orientée Objet tels que les designs patterns (MVC, Strategy), l'héritage... D'autre part, nous avons tenté de découper le code afin que le travail en équipe soit facilité dans le sens où la compréhension de chaque fonction ou classe soit la plus claire et plus rapide pour chaque membre du groupe.

%Section 3 : Les algorithmes
\section{Les algorithmes}

%Sub 1 : Random
\subsection{Déplacements aléatoires}

L'algorithme de déplacement alétoire récupère tout d'abord pour chaque personnage sa position actuelle sur le tableau. Ensuite, il récupère les cases voisines (droite/gauche et haut/bas) puis en sélectionne une au hasard. Si la case sélectionnée est un mur il rétière la sélection sinon il retourne la case sélectionnée pour que le programme effectue le déplacement.

%Sub 2 : Plus court chemin
\subsection{Algorithme du plus court chemin}


%Sub 3 : minimax
\subsection{Minimax}

L'implémentation de l'algorithme MiniMax fut très intéressante.\\
Elle permit de gagner en connaissance sur ce pilier de l'intelligence artificielle et ses propriétés.\\\\

Un des points forts de cette algorithme est de pouvoir retourner une position à $n$ personnages à chaque appel. C'est-à-dire que le coût sera le même pour déplacer Pacman seul ou bien tous les personnages présents sur la carte.\\\\

Un autre point intéressant est la gestion des paramètres. Ces-derniers sont notamment la profondeur maximale ainsi que les valeurs des fonctions d'évaluation.
Ainsi bien que des valeurs soient proposées par le sujet, étudier les changements de comportements en fonction des paramètres permit de comprendre les rapports entre les objectifs de chaque personnage (ne pas être mangé, manger une super pacgomme, ...).\\\\


Passons maintenant à la justification de l'implémentation.\\
Pour commencer aucune structure d'arbre n'a été utilisée. En effet les données calculées ne sont pas perstistantes et ne nécessitent donc pas de structure pour les sauvegarder.\\
L'algorithme est donc une suite d'appels récursifs entre deux fonctions : fonctionMax et fonctionMin. Ces fonctions, simulant respectivement un tour d'un agent Max(PacMan) et d'un agent Min(Ghost). Un tour correspond donc à un appel à fonctionMax pour chaque agentMax et à fonctionMin pour chaque agentMin.\\
Pendant le déroulement d'un tour un score commun à tous les agents est utilisé. C'est la valeur retournée par fonctionMax et fonctionMin. Le but d'un agent Max est de maximiser ce score, un agent Min cherchera lui à le minimiser.\\
Les appels se font tant que la profondeur maximale n'est pas atteinte.\\
Une fois atteinte, les valeurs remontent et les choix de Min et Max se font. La meilleure destination pour chaque agent est choisie à ce moment.\\\\

Deux fonctions getDestinationPacman et getDestinationGhost permettent d'acceder à ces destinations. Si la destination est déjà calculée on appelle Minimax puis on retourne la valeur et on lui affecte null (pour que Minimax soit appelé l'itération suivante).
Les attributs contenant ces destinations sont statiques afin d'être communs à tous les personnages possédant cette stratégie.\\\\

Bien que l'implementation actuelle ne traite qu'un seul agent Max et qu'un seul agent Min, la généralisation à n peut se faire facilement via l'utilisation de Collection et de simple itérateurs.\\\\

Voilà qui conclus cette étude de MiniMax.

\end{document}

